\section{Pengantar Negative Feedback}

\subsection{Empat Jenis Negative Feedback}	
\begin{frame}{Empat Jenis Negative Feedback}	
	\begin{itemize}
		\item Negative feeback pertama $ \rightarrow $ menstabilkan voltage gain, meningkatkan impedansi input, menurunkan impedansi output
		\item Dengan adanya transistor \& op amps $\rightarrow$ bertambah 3 jenis negative feedback
	\end{itemize}
\end{frame}

\subsection{Ide Dasar}
\begin{frame}{Ide Dasar}
	\begin{itemize}
		\item Input dan output negative feedback amplifier bisa berupa tegangan maupun arus
	\end{itemize}
\end{frame}

\subsection{Konverter}
\begin{frame}{Konverter}
	content...
\end{frame}

\subsection{Diagram}
\begin{frame}{Diagram}
	content...
\end{frame}