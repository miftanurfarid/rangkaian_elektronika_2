\section{Inverting Amplifier}

%%%%%%%%%%%%%%%%%%%%%%%%%%%%%%%%%%%%%%%%%%
\subsection{Pengantar Inverting Amplifier}

\begin{frame}[t]{Pengantar Inverting Amplifier}
	\begin{itemize}
		\item Inverting amplifier: rangkaian op amp paling dasar.
		\item Menggunakan negative feedback untuk menstabilkan keseluruhan voltage gain.
		\item Keseluruhan voltage gain perlu distabilkan karena $ A_{VOL} $ sangat besar dan tidak stabil untuk digunakan tanpa feedback.
		\item Contohnya, 741C memiliki $ A_{VOL} $ minimum sebesar 20.000 dan $ A_{VOL} $ maksimum lebih dari 200.000.
		\item Voltage gain yang tidak dapat dipredisksi dari magnitude dan variasi ini tidak berguna tanpa feedback.
	\end{itemize}
\end{frame}

\subsection{Inverting Negative Feedback}
\begin{frame}[t]{Inverting Negative Feedback}
	\begin{figure}
		\centering
		\includegraphics[height=0.5\textheight]{gambar/fig-16.12}
		\caption{Inverting amplifier}
		\label{fig-16.12}
	\end{figure}
\end{frame}

\begin{frame}[t]{Inverting Negative Feedback}
	\begin{itemize}
		\item Gambar \ref{fig-16.12} menunjukkan sebuah inverting amplifier.
		\item Tegangan power-supply tidak ditampilkan agar gambar lebih sederhana.
		\item Tegangan input $ v_{in} $ men-drive inverting input melalui resistor $ R_1 $.
		\item Menghasilkan tegangan inverting output $ v_2 $.
		\item Tegangan input dikuatkan oleh open-loop voltage gain untuk menghasilkan tegangan inverted output.
		\item Tegangan output diumpanbalik ke input melalui resistor feedback $ R_f $.
		\item Menghasilkan negative feedback karena output memiliki beda fasa sebesar 180$ ^\circ $ dengan input.
		\item Dengan kata lain, setiap perubahan di $ v_2 $ yang dihasilkan oleh tegangan input berkebalikan dengan sinyal output.
	\end{itemize}
\end{frame}

\begin{frame}[t]{Inverting Negative Feedback}
	\begin{itemize}
		\item Bagaimana negative feedback dapat menstabilkan overall voltage gain?
		\item Jika open-loop voltage gain $ A_{VOL} $ meningkat dengan alasan apa pun, tegangan output akan meningkat dan memberikan tegangan feedback yang lebih banyak ke inverting input.
		\item Tegangan feedback yang berkebalikan ini akan mereduksi tegangan $ v_2 $.
		\item Karena itu, meskipun $ A_{VOL} $ meningkat,$ v_2 $ menurun, dan output akhir meningkat jauh lebih sedikit daripada tanpa negative feedback.
	\end{itemize}
\end{frame}

%%%%%%%%%%%%%%%%%%%%%%%%%%%%%%%%%%%%%%%%%%
\subsection{Virtual Ground}

\begin{frame}[t]{Virtual Ground}
	\begin{itemize}
		\item Jika kita menghubungkan kabel antara suatu titik di dalam rangkaian ke ground, tegangan pada titik tersebut menjadi nol.
		\item Kabel tersebut memberikan jalur untuk arus mengalir ke ground.
		\item Mechanical ground (kabel yang menghubungkan titik ke ground) adalah ground untuk tegangan dan arus.
		\item Lain halnya dengan Virtual Ground, salah satu jenis ground yang digunakan untuk menganalisis inverting amplifier dengan lebih mudah.
	\end{itemize}
\end{frame}

\begin{frame}[t]{Virtual Ground}
	\begin{itemize}
		\item Konsep dari virtual ground berdasarkan op amp ideal.
		\item Op amp ideal memiliki open-loop voltage gain yang tak berhingga dan resistansi input tak berhingga.
		\item Karenanya, kita dapat menyimpulkan sebagai berikut (Gambar \ref{fig-16.13}):
		\begin{itemize}
			\item Karena $ I_{in} $ adalah tak berhingga maka $ i_2 $ adalah nol.
			\item Karena $ A_{VOL} $ adalah tak berhingga, maka $ v_2 $ adalah nol.
		\end{itemize}
	\end{itemize}
\end{frame}

\begin{frame}[t]{Virtual Ground}
	\begin{figure}
		\centering
		\includegraphics[height=0.7\textheight]{gambar/fig-16.13}
		\caption{Konsep virtual ground}
		\label{fig-16.13}
	\end{figure}
\end{frame}

\begin{frame}[t]{Virtual Ground}
	\begin{itemize}
		\item Karena $ i_2 $ = nol, arus yang melalui $ R_f $ pasti sama dengan arus input yang melalui $ R_1 $.
		\item Karena $ v_2 $ = nol, virtual ground yang ditunjukkan pada Gambar \ref{fig-16.13} menunjukkan bahwa inverting input bertindak seperti ground untuk tegangan dan open untuk arus.
	\end{itemize}
\end{frame}

%%%%%%%%%%%%%%%%%%%%%%%%%%%%%%%%%%%%%%%%%%
\subsection{Voltage Gain}

\begin{frame}[t]{Voltage Gain}
	\begin{itemize}
		\item Gambar \ref{fig-16.14} menunjukkan virtual ground pada inverting input.
		\item Sisi kanan dari $ R_1 $ adalah ground tegangan, maka dapat kita tulis:\\
		$$ v_{in} = i_{in} R_1 $$
		\item Begitu juga sisi kiri dari $ R_f $ adalah ground tegangan, sehingga magnitude dari tegangan output adalah:\\
		$$ v_{out} = -i_{in} R_f $$
		\item Untuk mendapatkan voltage gain, maka $ v_{out} $ dibagi dengan $ v_{in} $:
		\begin{equation}\label{pers.16.3}
			A_{v(CL)} = \frac{-R_f}{R_1}
		\end{equation}\\
		dimana $ A_{v(CL)} $ adalah closed-loop voltage gain.
	\end{itemize}
\end{frame}

\begin{frame}[t]{Voltage Gain}
	\begin{itemize}
		\item Disebut closed-loop voltage gain karena merupakan tegangan ketika terdapat jalur feedback antara output dan input.
		\item Karena negative feedback, closed-loop voltage gain, $ A_{v(CL)} $, selalu lebih kecil daripada open-loop voltage gain, $ A_{VOL} $.
		\item Misalkan, jika $ R_1 $ = 1 k$\Omega$ dan $ R_f $ = 50 k$\Omega$, maka closed-loop voltage gain sebesar 50.
		\item Tanda negatif pada Persamaan \ref{pers.16.3} menunjukkan voltage gain memiliki beda fasa sebesar 180$ ^\circ $.
	\end{itemize}
\end{frame}

\begin{frame}[t]{Voltage Gain}
	\begin{figure}
		\centering
		\includegraphics[height=0.4\textheight]{gambar/fig-16.14}
		\caption{Inverting amplifier memiliki arus yang sama yang melewati kedua resistor}
		\label{fig-16.14}
	\end{figure}
\end{frame}

%%%%%%%%%%%%%%%%%%%%%%%%%%%%%%%%%%%%%%%%%%
\subsection{Input Impedance}

\begin{frame}[t]{Input Impedance}
	\begin{itemize}
		\item Dalam beberapa hal, terkadang seorang engineer menginginkan impendansi input tertentu.
		\item Ini salah satu kelebihan dari inverting amplifier, mudah untuk mengatur impedansi input yang diinginkan.
		\item Karena sisi kanan dari $ R_1 $ adalah virtual ground, sehingga closed-loop input impedansinya adalah:\\

		\begin{equation}\label{pers.16.04}
			z_{in(CL)} = R_1
		\end{equation}

		\item Ini adalah impedansi input yang ada di sisi kiri dari $ R_1 $ seperti yang ditunjukkan oleh Gambar \ref{fig-16.14}.
		\item Misalkan, jika impedansi input sebesar 2 k$\Omega$ dan closed-loop voltage gain sebesar 50 yang dibutuhkan, maka engineer akan menggunakan $ R_1 $ = 2 k$\Omega$ dan $ R_f $ = 100 k$\Omega$.
	\end{itemize}
\end{frame}

%%%%%%%%%%%%%%%%%%%%%%%%%%%%%%%%%%%%%%%%%%
\subsection{Bandwidth}

\begin{frame}[t]{Bandwidth}
	\begin{itemize}
		\item Open-loop bandwidth atau frekuensi cut-off dari op amp sangat kecil.
		\item Disebabkan oleh internal compensating capacitor.
		\item Untuk 741C: $$ f_{2(OL)} = 10 \text{ Hz} $$
		\item Pada frekuensi ini, open-loop voltage gain akan berhenti dan turun dengan respon orde-1.
	\end{itemize}
\end{frame}

\begin{frame}[t]{Bandwidth}
	\begin{itemize}
		\item Ketika negative feedback digunakan, overall bandwidth akan meningkat.
		\item Karena, jika frekuensi input lebih besar daripada $ f_2(OL) $, $ A_{VOL} $ menurun sebesar 20 dB/decade.
		\item Ketika $ v_{out} $ mencoba untuk turun, tegangan yang berkebalikan akan diumpan-balik ke inverting input.
		\item Sehingga, $ v_2 $ meningkat dan mengkompensasi penurunan $ A_{VOL} $.
		\item Karena hal ini lah maka $ A_{v(CL)} $ berhenti pada frekuensi yang lebih besar daripada $ f_2(OL) $.
		\item Semakin besar negative feedback ($ A_{v(CL)} $ lebih kecil) maka closed-loop bandwidth $ f_2(CL) $ semakin besar.
	\end{itemize}
\end{frame}

\begin{frame}[t]{Bandwidth}
	\begin{itemize}
		\item Gambar \ref{fig-16.15} menunjukkan bagaimana closed-loop bandwidth meningkat dengan adanya negative feedback.
		\item Semakin besar negative feedback ($ A_{v(CL)} $ lebih kecil), semakin besar closed-loop bandwidth $ f_2(CL) $.

		\begin{equation*}
			f_{2(CL)} = \frac{f_{unity}}{A_{v(CL)} + 1}
		\end{equation*}
		
		\item Umumnya, $ A_v(CL) $ lebih besar daripada 10, sehingga persamaan di atas dapat disederhanakan menjadi
		
		\begin{equation}\label{pers.16.05}
			f_{2(CL)} = \frac{f_{unity}}{A_{v(CL)}}
		\end{equation}
	
		\begin{equation}\label{pers.16.06}
			f_{unity} = A_{v(CL)}f_{2(CL)}
		\end{equation}
	\end{itemize}
\end{frame}

%%%%%%%%%%%%%%%%%%%%%%%%%%%%%%%%%%%%%%%%%%
\subsection{Bias dan Offset}

\begin{frame}[t]{Bias dan Offset}
	\begin{itemize}
		\item Negative feedback mengurangi error output yang disebabkan oleh arus bias input, arus offset input, dan tegangan offset input.
		\item Seperti yang telah didiskusikan pada bab sebelumnya, ketika tegangan error input dan persamaan tegangan error output total adalah:\\

		\[ V_{error} = A_{VOL} (V_{1err} + V_{2err} + V_{3err}) \]

		\item Ketika negative feedback digunakan, persamaan di atas menjadi:

		\begin{equation}\label{pers.16.07}
			V_{error} \cong \pm A_{v(CL)} ( \pm V_{1err} \pm V_{2err} \pm V_{3err} )
		\end{equation}
		\\
		dimana $ V_{error} $ adalah tegangan error output total.
		\item Datasheet tidak menunjukkan tanda $ \pm $ karena hal ini dapat menunjukkan bahwa error bisa terjadi di kedua arah.
	\end{itemize}
\end{frame}

\begin{frame}[t]{Bias dan Offset}
	\begin{itemize}
		\item Error input sama seperti sebelumnya, yaitu:
		
		\begin{equation}\label{pers.16.08}
			V_{1err} = (R_{B1} - R_{(B2)}) I_{in(bias)}
		\end{equation}
	
		\begin{equation}\label{pers.16.09}
			V_{2err} = (R_{B1} + R_{(B2)}) \frac{I_{in(off)}}{2}
		\end{equation}
	
		\begin{equation}\label{pers.16.10}
			V_{3err} = V_{in(off)}
		\end{equation}
	
	\end{itemize}
\end{frame}

\begin{frame}[t]{Bias dan Offset}
	\begin{itemize}
		\item Saat $ A_{v(CL)} $ kecil, error output total yang diberikan dari Persamaan \ref{pers.16.07} mungkin cukup kecil untuk diabaikan.
		\item Jika tidak, resistor compensation dan offset nulling menjadi perlu.
		\item Di dalam inverting amplifier, $ R_{B2} $ adalah resistor Thevenin.
		\item Resistor Thevenin:
		
		\begin{equation}\label{pers.16.11}
			R_{B2} = R_1 \parallel R_f
		\end{equation}

		\item Jika perlu untuk mengkompensasi arus bias input, resistor $ R_{B1} $ yang bernilai sama dapat dihubungkan ke noninverting input.
		\item Resistor ini tidak berdampak pada teknik virtual ground karena tidak ada arus sinyal AC yang akan mengalir melaluinya.
	\end{itemize}
\end{frame}


%%%%%%%%%%%%%%%%%%%%%%%%%%%%%%%%%%%%%%%%%%

\subsection{Contoh Soal 2.7}

\begin{frame}[t]{Contoh Soal 2.7}
	\begin{multicols}{2}
		\begin{center}
			\includegraphics[width=0.7\linewidth]{gambar/fig-16.16a}
		\end{center}
		\columnbreak
		\begin{itemize}
			\item Pertanyaan:
			\begin{itemize}
				\item Berapa closed-loop voltage gain dan closed-loop bandwidth nya?
				\item Berapa tegangan output di 1 kHz? dan di 1 MHz?
			\end{itemize}
		\end{itemize}
	\end{multicols}
\end{frame}

\begin{frame}[t]{Contoh Soal 2.7}
	\begin{multicols}{2}
		\begin{center}
			\includegraphics[width=0.7\linewidth]{gambar/fig-16.16b}
		\end{center}
		\columnbreak
		\begin{itemize}
			\item Jawaban:
			\begin{itemize}
				\item Closed-loop voltage gain:
				\[ A_{v(CL)} = \frac{-R_f}{R_1} = \frac{-75 \text{ k}\Omega}{1.5 \text{ k}\Omega} = -50\]
				\item Closed-loop bandwidth:
				\[ f_{2(CL)} = \frac{f_{unity}}{A_{v(CL)}} = \frac{1 \text{ MHz}}{50} = 20 \text{ kHz}\]
				\item Gambar di samping adalah ideal bode-plot dari closed-loop voltage gain, $ A_{v(CL)} $.
				\item $ A_{v(CL)} = 20 \times \log(50) = 34 \text{ dB} $
			\end{itemize}
		\end{itemize}
	\end{multicols}
\end{frame}

\begin{frame}[t]{Contoh Soal 2.7}
	\begin{multicols}{2}
		\begin{center}
			\includegraphics[width=0.7\linewidth]{gambar/fig-16.16b}
		\end{center}
		\columnbreak
		\begin{itemize}
			\item Jawaban:
			\begin{itemize}
				\item Tegangan output di 1 kHz:
				\begin{align*}
					v_{out} &= (A_{v(CL)})(v_{in}) = (-50)(10 \text{ mVp-p}) \\
					&= -500 \text{ mVp-p}
				\end{align*}
				\item Tegangan output di 1 MHz. Karena 1 MHz adalah unity-gain frekuensinya, maka
					\[ v_{out} = -10 \text{ mVp-p} \]
				\item Tanda negatif menunjukkan phase-shift $ 180^{\circ} $ antara input dan output
			\end{itemize}
		\end{itemize}
	\end{multicols}
\end{frame}

\subsection{Latihan Soal 2.7}
\begin{frame}[t]{Latihan Soal 2.7}
	\begin{multicols}{2}
		\begin{center}
			\includegraphics[width=\linewidth]{gambar/fig-16.16a}
		\end{center}
		\columnbreak
		\begin{itemize}
			\item Pertanyaan:
			\begin{itemize}
				\item Berapa tegangan output di 100 kHz ?
				\item \textit{Hint:} Gunakan Persamaan \[ A_v = \frac{A_{v(mid)}}{ \sqrt{1 + (f/f_2)^2} } \]
			\end{itemize}
		\end{itemize}
	\end{multicols}
\end{frame}

\subsection{Contoh Soal 2.8}
\begin{frame}[t]{Contoh Soal 2.8}
	\begin{multicols}{2}
		\begin{center}
			\includegraphics[width=\linewidth]{gambar/fig-16.17a}
		\end{center}
		\columnbreak
		\begin{itemize}
			\item Pertanyaan:
			\begin{itemize}
				\item Berapa tegangan output ketika $ v_{in} = 0 $?
			\end{itemize}
		\end{itemize}
	\end{multicols}
\end{frame}

\begin{frame}[t]{Contoh Soal 2.8}
	\begin{multicols}{2}
		\begin{center}
			\includegraphics[width=\linewidth]{gambar/fig-16.17a}
		\end{center}
		\columnbreak
		\begin{itemize}
			\item Jawaban:
			\begin{itemize}
				\item Berdasarkan Tabel di Gambar \ref{tab-16.01}, didapatkan:
				\[ I_{\text{in}(\text{bias})} = 80 \text{ nA} \]
				\[ I_{\text{in}(\text{off})} = 20 \text{ nA} \]
				\[ V_{\text{in}(\text{off})} = 2 \text{ mV} \]
				\item Berdasarkan Persamaan \ref{pers.16.11}:
				\begin{align*}
					R_{B2} &= R_1 \parallel R_f = 1.5 \text{ k}\Omega \parallel 75 \text{ k}\Omega \\
					&= 1.47 \text{ k}\Omega
				\end{align*}
			\end{itemize}
		\end{itemize}
	\end{multicols}
\end{frame}

\begin{frame}[t]{Contoh Soal 2.8}
	\begin{multicols}{2}
		\begin{center}
			\includegraphics[width=\linewidth]{gambar/fig-16.17a}
		\end{center}
		\columnbreak
		\begin{itemize}
			\item Jawaban:
			\begin{itemize}
				\item Karena menggunakan analisis virtual ground, maka $ R_{B1} $ tidak berpengaruh apa-apa.
				\item Sehingga ketiga tegangan error input:
				\begin{align*}
					V_{1err} &= (R_{B1} - R_{B2})I_{in(bias)} \\
					&= ( - 1.47 \text{ k}\Omega )(80 \text{ nA}) \\
					&= -0.118 \text{ mV} \\
					V_{2err} &= (R_{B1} + R_{B2}) \frac{I_{in(off)}}{2} \\
					&= ( 1.47 \text{ k}\Omega )(10 \text{ nA}) \\
					&= 0.0147 \text{ mV} \\
				\end{align*}
			\end{itemize}
		\end{itemize}
	\end{multicols}
\end{frame}

\begin{frame}[t]{Contoh Soal 2.8}
	\begin{multicols}{2}
		\begin{center}
			\includegraphics[width=\linewidth]{gambar/fig-16.17a}
		\end{center}
		\columnbreak
		\begin{itemize}
			\item Jawaban:
			\begin{itemize}
				\item Sehingga ketiga tegangan error input (lanjutan):
				
				\begin{align*}
					V_{3err} &= V_{in(off)} = 2 \text{ mV}
				\end{align*}
				
				\item Closed-loop voltage gain:
				
				\[ A_{v(CL)} = \frac{-R_f}{R_1} = \frac{-75 \text{ k}\Omega}{1.5 \text{ k}\Omega} = -50\]
				
			\end{itemize}
		\end{itemize}
	\end{multicols}
\end{frame}

\begin{frame}[t]{Contoh Soal 2.8}
	\begin{multicols}{2}
		\begin{center}
			\includegraphics[width=\linewidth]{gambar/fig-16.17a}
		\end{center}
		\columnbreak
		\begin{itemize}
			\item Jawaban:
			\begin{itemize}
				
				\item Error tegangan output:
				\begin{align*}
					V_{error} &= \pm 50 (V_{1err} + V_{2err} + V_{2err}) \\
					&= \pm 50 (0.118 \text{mV} + 0.0147 \text{mV} + 2 \text{mV}) \\
					&= \pm 107 \text{ mV}
				\end{align*}
			\end{itemize}
		\end{itemize}
	\end{multicols}
\end{frame}

\subsection{Latihan Soal 2.8}
\begin{frame}[t]{Latihan Soal 2.8}
	\begin{multicols}{2}
		\begin{center}
			\includegraphics[width=\linewidth]{gambar/fig-16.17a}
		\end{center}
		\columnbreak
		\begin{itemize}
			\item Pertanyaan:
			\begin{itemize}
				\item Jika op amp yang digunakan adalah LF157A, berapa tegangan output ketika $ v_{in} = 0 $?
			\end{itemize}
		\end{itemize}
	\end{multicols}
\end{frame}

\subsection{Contoh Soal 2.9}
\begin{frame}[t]{Contoh Soal 2.9}
	\begin{multicols}{2}
		\begin{center}
			\includegraphics[width=\linewidth]{gambar/fig-16.17a}
		\end{center}
		\columnbreak
		\begin{itemize}
			\item Pertanyaan:
			\begin{itemize}
				\item Diketahui: \\
				$ I_{in(bias)} = 500 \text{ nA} $, \\
				$ I_{in(off)} = 200 \text{ nA}$, dan \\
				$ V_{in(off)} = 6 \text{ mV} $
				\item Berapa tegangan output jika $ v_{in} = 0 $ ?
			\end{itemize}
		\end{itemize}
	\end{multicols}
\end{frame}

\begin{frame}[t]{Contoh Soal 2.9}
	\begin{multicols}{2}
		\begin{center}
			\includegraphics[width=\linewidth]{gambar/fig-16.17a}
		\end{center}
		\columnbreak
		\begin{itemize}
			\item Jawaban:
			\begin{itemize}
				\item Tegangan error input:
				\begin{align*}
					V_{1err} &= (R_{B1} - R_{B2})I_{in(bias)} \\
					&= ( - 1.47 \text{ k}\Omega )(500 \text{ nA}) \\
					&= -0.735 \text{ mV} \\
					V_{2err} &= (R_{B1} + R_{B2}) \frac{I_{in(off)}}{2} \\
					&= ( 1.47 \text{ k}\Omega )(100 \text{ nA}) \\
					&= 0.147 \text{ mV} \\
					V_{3err} &= V_{in(off)} = 6 \text{ mV}
				\end{align*}
			\end{itemize}
		\end{itemize}
	\end{multicols}
\end{frame}

\begin{frame}[t]{Contoh Soal 2.9}
	\begin{multicols}{2}
		\begin{center}
			\includegraphics[width=\linewidth]{gambar/fig-16.17a}
		\end{center}
		\columnbreak
		\begin{itemize}
			\item Jawaban:
			\begin{itemize}
				\item Closed-loop voltage gain:
				\[ A_{v(CL)} = \frac{-R_f}{R_1} = \frac{-75 \text{ k}\Omega}{1.5 \text{ k}\Omega} = -50\]
				\item Tegangan error output:
				\begin{align*}
					V_{error} &= \pm 50 (V_{1err} + V_{2err} + V_{2err}) \\
					&= \pm 50 (0.735 \text{mV} + 0.147 \text{mV} + 6 \text{mV}) \\
					&= \pm 344 \text{ mV}
				\end{align*}
			\end{itemize}
		\end{itemize}
	\end{multicols}
\end{frame}

\begin{frame}[t]{Contoh Soal 2.9}
	\begin{itemize}
		\item Pada Contoh Soal 2.7, tegangan ouput yang diinginkan adalah 500 mVp-p.
		\item Bisakah kita mengabaikan tegangan error output yang besar?
		\item Terganti dalam pengaplikasiannya.
		\item Untuk amplifier sinyal suara, kita hanya butuh di rentang frekuensi suara yaitu 20 Hz hingga 20 kHz.
		\item Maka kita secara kapasitif memasangkan output ke resistor beban atau tahap berikutnya.
		\item Hal ini akan mem-blok tegangan error output DC tapi tetap mentransmisikan sinyal AC. Sehingga Error output menjadi tidak relevan lagi.
	\end{itemize}
\end{frame}

\begin{frame}[t]{Contoh Soal 2.9}
	\begin{itemize}
		\item Jika kita ingin menguatkan sinyal dengan frekuensi dari 0 Hz higga 20 kHz, maka kita membutuhkan op amp yang lebih baik (bias dan offset yang lebih kecil), atau memodifikasi rangkaiannya seperti pada Gambar \ref{fig-16.17}b.
		\item Pada Gambar \ref{fig-16.17}b, kita menambahkan compensating resistor terhadap noinverting input untuk menghilangkan efek dari arus bias input.
		\item Digunakan juga potentiometer 10 k$\Omega$ untuk meniadakan efek dari arus offset input dan tegangan offset input.
	\end{itemize}
\end{frame}

\begin{frame}[t]{Contoh Soal 2.9}
	\begin{figure}
		\centering
		\includegraphics[width=0.8\linewidth]{gambar/fig-16.17}
		\caption{(a) Rangkaian op amp 741C dan (b) Rangkaian op amp 741C dengan penambahan compensating resistor dan potensiometer}
		\label{fig-16.17}
	\end{figure}
	
\end{frame}