\section{Inverting Amplifier}

\subsection{Pengantar Inverting Amplifier}
\begin{frame}{Pengantar Inverting Amplifier}
	\begin{itemize}
		\item Inverting amplifier: rangkaian op amp paling dasar.
		\item Menggunakan negative feedback untuk menstabilkan keseluruhan voltage gain.
		\item Keseluruhan voltage gain perlu distabilkan karena $ A_{VOL} $ sangat besar dan tidak stabil untuk digunakan tanpa feedback.
		\item Contohnya, 741C memiliki $ A_{VOL} $ minimum sebesar 20.000 dan $ A_{VOL} $ maksimum lebih dari 200.000.
		\item Voltage gain yang tidak dapat dipredisksi dari magnitude dan variasi ini tidak berguna tanpa feedback.
	\end{itemize}
\end{frame}

\subsection{Inverting Negative Feedback}
\begin{frame}{Inverting Negative Feedback}
	\begin{figure}
		\centering
		\includegraphics[height=0.5\textheight]{gambar/fig-16.12}
		\caption{Inverting amplifier}
		\label{fig-16.12}
	\end{figure}
\end{frame}

\begin{frame}{Inverting Negative Feedback}
	\begin{itemize}
		\item Gambar \ref{fig-16.12} menunjukkan sebuah inverting amplifier.
		\item Tegangan power-supply tidak ditampilkan agar gambar lebih sederhana.
		\item Tegangan input $ v_{in} $ men-drive inverting input melalui resistor $ R_1 $.
		\item Menghasilkan tegangan inverting output $ v_2 $.
		\item Tegangan input dikuatkan oleh open-loop voltage gain untuk menghasilkan tegangan inverted output.
		\item Tegangan output diumpanbalik ke input melalui resistor feedback $ R_f $.
		\item Menghasilkan negative feedback karena output memiliki beda fasa sebesar 180$ ^\circ $ dengan input.
		\item Dengan kata lain, setiap perubahan di $ v_2 $ yang dihasilkan oleh tegangan input berkebalikan dengan sinyal output.
	\end{itemize}
\end{frame}

\begin{frame}{Inverting Negative Feedback}
	\begin{itemize}
		\item Bagaimana negative feedback dapat menstabilkan overall voltage gain?
		\item Jika open-loop voltage gain $ A_{VOL} $ meningkat dengan alasan apa pun, tegangan output akan meningkat dan memberikan tegangan feedback yang lebih banyak ke inverting input.
		\item Tegangan feedback yang berkebalikan ini akan mereduksi tegangan $ v_2 $.
		\item Karena itu, meskipun $ A_{VOL} $ meningkat,$ v_2 $ menurun, dan output akhir meningkat jauh lebih sedikit daripada tanpa negative feedback.
	\end{itemize}
\end{frame}

\subsection{Virtual Ground}

\begin{frame}{Virtual Ground}
	\begin{itemize}
		\item Jika kita menghubungkan kabel antara suatu titik di dalam rangkaian ke ground, tegangan pada titik tersebut menjadi nol.
		\item Kabel tersebut memberikan jalur untuk arus mengalir ke ground.
		\item Mechanical ground (kabel yang menghubungkan titik ke ground) adalah ground untuk tegangan dan arus.
		\item Lain halnya dengan Virtual Ground, salah satu jenis ground yang digunakan untuk menganalisis inverting amplifier dengan lebih mudah.
	\end{itemize}
\end{frame}

\begin{frame}{Virtual Ground}
	\begin{itemize}
		\item Konsep dari virtual ground berdasarkan op amp ideal.
		\item Op amp ideal memiliki open-loop voltage gain yang tak berhingga dan resistansi input tak berhingga.
		\item Karenanya, kita dapat menyimpulkan sebagai berikut (Gambar \ref{fig-16.13}):
		\begin{itemize}
			\item Karena $ I_{in} $ adalah tak berhingga maka $ i_2 $ adalah nol.
			\item Karena $ A_{VOL} $ adalah tak berhingga, maka $ v_2 $ adalah nol.
		\end{itemize}
	\end{itemize}
\end{frame}

\begin{frame}{Virtual Ground}
	\begin{figure}
		\centering
		\includegraphics[height=0.7\textheight]{gambar/fig-16.13}
		\caption{Konsep virtual ground}
		\label{fig-16.13}
	\end{figure}
\end{frame}

\begin{frame}{Virtual Ground}
	\begin{itemize}
		\item Karena $ i_2 $ = nol, arus yang melalui $ R_f $ pasti sama dengan arus input yang melalui $ R_1 $.
		\item Karena $ v_2 $ = nol, virtual ground yang ditunjukkan pada Gambar \ref{fig-16.13} menunjukkan bahwa inverting input bertindak seperti ground untuk tegangan dan open untuk arus.
	\end{itemize}
\end{frame}

\subsection{Voltage Gain}
\begin{frame}{Voltage Gain}
	\begin{itemize}
		\item Gambar \ref{fig-16.14} menunjukkan virtual ground pada inverting input.
		\item Sisi kanan dari $ R_1 $ adalah ground tegangan, maka dapat kita tulis:\\
		$$ v_{in} = i_{in} R_1 $$
		\item Begitu juga sisi kiri dari $ R_f $ adalah ground tegangan, sehingga magnitude dari tegangan output adalah:\\
		$$ v_{out} = -i_{in} R_f $$
		\item Untuk mendapatkan voltage gain, maka $ v_{out} $ dibagi dengan $ v_{in} $:
		\begin{equation}\label{pers.16.3}
			A_{v(CL)} = \frac{-R_f}{R_1}
		\end{equation}\\
		dimana 
	\end{itemize}
\end{frame}

\begin{frame}{Voltage Gain}
	\begin{figure}
		\centering
		\includegraphics[height=0.7\textheight]{gambar/fig-16.14}
		\caption{Inverting amplifier memiliki arus yang sama yang melewati kedua resistor}
		\label{fig-16.14}
	\end{figure}
\end{frame}

\subsection{Bandwidth}
\begin{frame}{Bandwidth}
	\begin{multicols}{2}
		\begin{figure}
			\centering
			\includegraphics[height=0.6\textheight]{gambar/fig-16.15}
			\caption{Voltage gain yang lebih kecil menghasilkan bandwidth yang lebih besar}
			\label{fig-16.15}
		\end{figure}
	\columnbreak
		\begin{itemize}
			\item Closed-loop bandwidth:
			\begin{equation}\label{pers.16.5}
				f_{2(CL)} = \frac{f_{unity}}{A_{v(CL)}}
			\end{equation}
			\item Gain-band-width product (GBW):
			\begin{equation}\label{pers.16.6}
				f_{unity} = A_{v(CL)}f_{2(CL)}
			\end{equation}
		\end{itemize}
	\end{multicols}
\end{frame}

\subsection{Bias dan Offset}
\begin{frame}{Bias dan Offset}
	\begin{itemize}
		\item Total error tegangan output:
		\begin{equation}\label{pers.16.7}
			V_{error} \cong \pm A_{v(CL)} (\pm V_{1err} \pm V_{2err} \pm V_{3err} )
		\end{equation}
		\item Error tegangan input:
		\begin{equation}\label{pers.16.8}
			V_{1err} = (R_{B1} - R_{(B2)}) I_{in(bias)}
		\end{equation}
		\begin{equation}\label{pers.16.9}
			V_{2err} = (R_{B1} + R_{(B2)}) \frac{I_{in(off)}}{2}
		\end{equation}
		\begin{equation}\label{pers.16.10}
			V_{3err} = V_{in(off)}
		\end{equation}
		\item Resistor Thevenin
		\begin{equation}\label{pers.16.11}
			R_{B2} = R_1 \parallel R_f
		\end{equation}
	\end{itemize}
\end{frame}

\subsection{Contoh Soal 2.7}
\begin{frame}{Contoh Soal 2.7}
	\begin{multicols}{2}
		\begin{center}
			\includegraphics[width=\linewidth]{gambar/fig-16.16a}
		\end{center}
		\columnbreak
		\begin{itemize}
			\item Pertanyaan:
			\begin{itemize}
				\item Berapa penguatan tegangan closed-loop dan bandwidth closed-loop nya?
				\item Berapa tegangan output di 1 kHz? dan di 1 MHz?
			\end{itemize}
		\end{itemize}
	\end{multicols}
\end{frame}

\begin{frame}{Contoh Soal 2.7}
	\begin{multicols}{2}
		\begin{center}
			\includegraphics[width=\linewidth]{gambar/fig-16.16b}
		\end{center}
		\columnbreak
		\begin{itemize}
			\item Jawaban:
			\begin{itemize}
				\item Penguatan tegangan closed-loop:
				\[ A_{v(CL)} = \frac{-R_f}{R_1} = \frac{-75 \text{ k}\Omega}{1.5 \text{ k}\Omega} = -50\]
				\item Bandwidth closed-loop:
				\[ f_{2(CL)} = \frac{f_{unity}}{A_{v(CL)}} = \frac{1 \text{ MHz}}{50} = 20 \text{ kHz}\]
				\item Ideal bode-plot dari $ A_{v(CL)} $
			\end{itemize}
		\end{itemize}
	\end{multicols}
\end{frame}

\begin{frame}{Contoh Soal 2.7}
	\begin{multicols}{2}
		\begin{center}
			\includegraphics[width=\linewidth]{gambar/fig-16.16b}
		\end{center}
		\columnbreak
		\begin{itemize}
			\item Jawaban:
			\begin{itemize}
				\item Tegangan output di 1 kHz:
				\begin{align*}
					v_{out} &= (A_{v(CL)})(v_{in}) = (-50)(10 \text{ mVp-p}) \\
					&= -500 \text{ mVp-p}
				\end{align*}
				\item Tegangan output di 1 MHz. Karena 1 MHz adalah unity-gain frekuensinya, maka
					\[ v_{out} = -10 \text{ mVp-p} \]
				\item Tanda negatif menunjukkan phase-shift $ 180^{\circ} $ antara input dan output
			\end{itemize}
		\end{itemize}
	\end{multicols}
\end{frame}

\subsection{Latihan Soal 2.7}
\begin{frame}{Latihan Soal 2.7}
	\begin{multicols}{2}
		\begin{center}
			\includegraphics[width=\linewidth]{gambar/fig-16.16a}
		\end{center}
		\columnbreak
		\begin{itemize}
			\item Pertanyaan:
			\begin{itemize}
				\item Berapa tegangan output di 100 kHz ?
				\item \textit{Hint:} Gunakan persamaan \[ A_v = \frac{A_{v(mid)}}{ \sqrt{1 + (f/f_2)^2} } \]
			\end{itemize}
		\end{itemize}
	\end{multicols}
\end{frame}

\subsection{Contoh Soal 2.8}
\begin{frame}{Contoh Soal 2.8}
	\begin{multicols}{2}
		\begin{center}
			\includegraphics[width=\linewidth]{gambar/fig-16.17a}
		\end{center}
		\columnbreak
		\begin{itemize}
			\item Pertanyaan:
			\begin{itemize}
				\item Berapa tegangan output ketika $ v_{in} = 0 $?
			\end{itemize}
		\end{itemize}
	\end{multicols}
\end{frame}

\begin{frame}{Contoh Soal 2.8}
	\begin{multicols}{2}
		\begin{center}
			\includegraphics[width=\linewidth]{gambar/fig-16.17a}
		\end{center}
		\columnbreak
		\begin{itemize}
			\item Jawaban:
			\begin{itemize}
				\item Berdasarkan Tabel di Gambar \ref{tab-16.01}, didapatkan:
				\[ I_{\text{in}(\text{bias})} = 80 \text{ nA} \]
				\[ I_{\text{in}(\text{off})} = 20 \text{ nA} \]
				\[ V_{\text{in}(\text{off})} = 2 \text{ mV} \]
				\item Berdasarkan Persamaan \ref{pers.16.11}:
				\begin{align*}
					R_{B2} &= R_1 \parallel R_f = 1.5 \text{ k}\Omega \parallel 75 \text{ k}\Omega \\
					&= 1.47 \text{ k}\Omega
				\end{align*}
			\end{itemize}
		\end{itemize}
	\end{multicols}
\end{frame}

\begin{frame}{Contoh Soal 2.8}
	\begin{multicols}{2}
		\begin{center}
			\includegraphics[width=\linewidth]{gambar/fig-16.17a}
		\end{center}
		\columnbreak
		\begin{itemize}
			\item Jawaban:
			\begin{itemize}
				\item Error tegangan input:
				\begin{align*}
					V_{1err} &= (R_{B1} - R_{B2})I_{in(bias)} \\
					&= ( - 1.47 \text{ k}\Omega )(80 \text{ nA}) \\
					&= -0.118 \text{ mV} \\
					V_{2err} &= (R_{B1} + R_{B2}) \frac{I_{in(off)}}{2} \\
					&= ( 1.47 \text{ k}\Omega )(10 \text{ nA}) \\
					&= 0.0147 \text{ mV} \\
					V_{3err} &= V_{in(off)} = 2 \text{ mV}
				\end{align*}
			\end{itemize}
		\end{itemize}
	\end{multicols}
\end{frame}

\begin{frame}{Contoh Soal 2.8}
	\begin{multicols}{2}
		\begin{center}
			\includegraphics[width=\linewidth]{gambar/fig-16.17a}
		\end{center}
		\columnbreak
		\begin{itemize}
			\item Jawaban:
			\begin{itemize}
				\item Penguatan tegangan closed-loop:
				\[ A_{v(CL)} = \frac{-R_f}{R_1} = \frac{-75 \text{ k}\Omega}{1.5 \text{ k}\Omega} = -50\]
				\item Error tegangan output:
				\begin{align*}
					V_{error} &= \pm 50 (V_{1err} + V_{2err} + V_{2err}) \\
					&= \pm 50 (0.118 \text{mV} + 0.0147 \text{mV} + 2 \text{mV}) \\
					&= \pm 107 \text{ mV}
				\end{align*}
			\end{itemize}
		\end{itemize}
	\end{multicols}
\end{frame}

\subsection{Latihan Soal 2.8}
\begin{frame}{Latihan Soal 2.8}
	\begin{multicols}{2}
		\begin{center}
			\includegraphics[width=\linewidth]{gambar/fig-16.17a}
		\end{center}
		\columnbreak
		\begin{itemize}
			\item Pertanyaan:
			\begin{itemize}
				\item Jika op amp yang digunakan adalah LF157A, berapa tegangan output ketika $ v_{in} = 0 $?
			\end{itemize}
		\end{itemize}
	\end{multicols}
\end{frame}

\subsection{Contoh Soal 2.9}
\begin{frame}{Contoh Soal 2.9}
	\begin{multicols}{2}
		\begin{center}
			\includegraphics[width=\linewidth]{gambar/fig-16.17a}
		\end{center}
		\columnbreak
		\begin{itemize}
			\item Pertanyaan:
			\begin{itemize}
				\item Diketahui: \\
				$ I_{in(bias)} = 500 \text{ nA} $, \\
				$ I_{in(off)} = 200 \text{ nA}$, dan \\
				$ V_{in(off)} = 6 \text{ mV} $
				\item Berapa tegangan output jika $ v_{in} = 0 $ ?
			\end{itemize}
		\end{itemize}
	\end{multicols}
\end{frame}

\begin{frame}{Contoh Soal 2.9}
	\begin{multicols}{2}
		\begin{center}
			\includegraphics[width=\linewidth]{gambar/fig-16.17a}
		\end{center}
		\columnbreak
		\begin{itemize}
			\item Jawaban:
			\begin{itemize}
				\item Error tegangan input:
				\begin{align*}
					V_{1err} &= (R_{B1} - R_{B2})I_{in(bias)} \\
					&= ( - 1.47 \text{ k}\Omega )(500 \text{ nA}) \\
					&= -0.735 \text{ mV} \\
					V_{2err} &= (R_{B1} + R_{B2}) \frac{I_{in(off)}}{2} \\
					&= ( 1.47 \text{ k}\Omega )(100 \text{ nA}) \\
					&= 0.147 \text{ mV} \\
					V_{3err} &= V_{in(off)} = 6 \text{ mV}
				\end{align*}
			\end{itemize}
		\end{itemize}
	\end{multicols}
\end{frame}

\begin{frame}{Contoh Soal 2.9}
	\begin{multicols}{2}
		\begin{center}
			\includegraphics[width=\linewidth]{gambar/fig-16.17a}
		\end{center}
		\columnbreak
		\begin{itemize}
			\item Jawaban:
			\begin{itemize}
				\item Penguatan tegangan closed-loop:
				\[ A_{v(CL)} = \frac{-R_f}{R_1} = \frac{-75 \text{ k}\Omega}{1.5 \text{ k}\Omega} = -50\]
				\item Error tegangan output:
				\begin{align*}
					V_{error} &= \pm 50 (V_{1err} + V_{2err} + V_{2err}) \\
					&= \pm 50 (0.735 \text{mV} + 0.147 \text{mV} + 6 \text{mV}) \\
					&= \pm 344 \text{ mV}
				\end{align*}
			\end{itemize}
		\end{itemize}
	\end{multicols}
\end{frame}

\begin{frame}{Contoh Soal 2.9}
	\begin{figure}
		\centering
		\includegraphics[width=0.8\linewidth]{gambar/fig-16.17}
		\caption{(a) Rangkaian op amp 741C dan (b) Rangkaian op amp 741C dengan penambahan compensating resistor dan potensiometer}
		\label{fig-16.17}
	\end{figure}
	
\end{frame}