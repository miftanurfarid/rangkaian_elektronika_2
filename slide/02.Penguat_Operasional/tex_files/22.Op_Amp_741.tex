\section{Op Amp 741}
\begin{frame}{Op Amp 741}
	\begin{itemize}
		\item Monolitic amp $ \mu\text{A709} $ tahun 1965 oleh Fairchild Semiconductor
		\item $ \mu\text{A709} $ memiliki kekurangan $ \rightarrow $ dibuatlah $ \mu\text{A741} $
		\item Banyak manufaktur yang membuat $ \mu\text{A741} $:
		\begin{itemize}
			\item ON Semiconductor: MC1741
			\item Texas Instruments: LM741
			\item Analog Devices: AD741.
		\end{itemize}
		\item Istilah umumnya op amp 741
	\end{itemize}
\end{frame}

\subsection{Standar Industri}
\begin{frame}{Standar Industri}
	\begin{itemize}
		\item Beberapa versi: 741, 741A, 741C, 741E, dan 741N
		\item Bergantung pada karakteristiknya (voltage gain, temp. range, noise level, dll)
		\item 741C (C = \textit{Commercial grade}) $ \rightarrow $ sedikit lebih murah dan paling banyak digunakan
		\item $ A_{VOL} = 100000 $, $ z_{in} = 2 \text{ M}\Omega $, $ z_out = 75~\Omega $
	\end{itemize}
\end{frame}

\begin{frame}{Standar Industri}
	\begin{figure}
		\centering
		\includegraphics[width=0.7\linewidth]{gambar/fig-16.03}
		\caption{Op amp 741 pinouts (a) dual-in-line, (b) ceramic flatpak, (c) metal can}
		\label{fig-16.03}
	\end{figure}
\end{frame}

\subsection{Rangkaian Ekivalen dari Op Amp 741}
\begin{frame}{Rangkaian Ekivalen dari Op Amp 741}
	\begin{multicols}{2}
		\begin{figure}
			\centering
			\includegraphics[width=0.9\linewidth]{gambar/fig-16.04}
			\caption{Rangkaian ekivalen dari op amp 741}
			\label{fig-16.04}
		\end{figure}
		\columnbreak
		\begin{itemize}
			\item Input diff amp
			\item Final Stage
			\item Active Loading
			\item Frequency Compensation $ C_{in(M)} = (A_v + 1) C_c $
		\end{itemize}
		\begin{figure}
			\centering
			\includegraphics[width=0.5\linewidth]{gambar/fig-16.05}
			\caption{Bode plot $ A_{VOL} $ 741C ideal}
			\label{fig-16.05}
		\end{figure}
	\vfill\null
	\end{multicols}
\end{frame}

\subsection{Bias \& Offset}
\begin{frame}{Bias \& Offset}
	\begin{multicols}{2}
		\begin{figure}
			\centering
			\includegraphics[width=0.8\linewidth]{gambar/fig-16.06}
			\caption{Penggunaan compensation dan nulling 741C}
			\label{fig-16.06}
		\end{figure}
		\columnbreak
		\begin{itemize}
			\item Tidak ada input signal $ \rightarrow $ input bias dan offset $ \rightarrow $ error output
			\item Error output berkurang $ \leftarrow $ base resistor yang sama $ \rightarrow $ hanya menghilangkan arus bias tapi tidak arus offset dan tegangan offset
			\item Solusi: menggunakan rangkaian nulling di datasheet
		\end{itemize}
		\vfill\null
	\end{multicols}
\end{frame}

\subsection{CMRR, MPP, dan $A_{VOL}$}
\begin{frame}{CMRR, MPP, dan $A_{VOL}$}
	\begin{itemize}
		\item Common-Mode Rejection Ratio (CMRR), Maximum Peak-to-Peak Output (MPP), Voltage Gain ($A_{VOL}$)
	\end{itemize}
	\begin{figure}
		\centering
		\includegraphics[width=0.9\linewidth]{gambar/fig-16.07}
		\caption{Grafik (a) CMRR, (b) MPP, dan (c) $A_{VOL}$ dari 741C}
		\label{fig:fig-16}
	\end{figure}
	\vfill\null
\end{frame}